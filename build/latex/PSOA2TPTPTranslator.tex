% Generated by Sphinx.
\def\sphinxdocclass{report}
\documentclass[letterpaper,10pt,english]{sphinxmanual}
\usepackage[utf8]{inputenc}
\DeclareUnicodeCharacter{00A0}{\nobreakspace}
\usepackage[T1]{fontenc}
\usepackage{babel}
\usepackage{times}
\usepackage[Bjarne]{fncychap}
\usepackage{longtable}
\usepackage{sphinx}


\title{PSOA2TPTP Translator Proposal Documentation}
\date{December 17, 2011}
\release{1.0.0}
\author{Reuben Peter-Paul, Gen Zou}
\newcommand{\sphinxlogo}{}
\renewcommand{\releasename}{Release}
\makeindex

\makeatletter
\def\PYG@reset{\let\PYG@it=\relax \let\PYG@bf=\relax%
    \let\PYG@ul=\relax \let\PYG@tc=\relax%
    \let\PYG@bc=\relax \let\PYG@ff=\relax}
\def\PYG@tok#1{\csname PYG@tok@#1\endcsname}
\def\PYG@toks#1+{\ifx\relax#1\empty\else%
    \PYG@tok{#1}\expandafter\PYG@toks\fi}
\def\PYG@do#1{\PYG@bc{\PYG@tc{\PYG@ul{%
    \PYG@it{\PYG@bf{\PYG@ff{#1}}}}}}}
\def\PYG#1#2{\PYG@reset\PYG@toks#1+\relax+\PYG@do{#2}}

\def\PYG@tok@gd{\def\PYG@tc##1{\textcolor[rgb]{0.63,0.00,0.00}{##1}}}
\def\PYG@tok@gu{\let\PYG@bf=\textbf\def\PYG@tc##1{\textcolor[rgb]{0.50,0.00,0.50}{##1}}}
\def\PYG@tok@gt{\def\PYG@tc##1{\textcolor[rgb]{0.00,0.25,0.82}{##1}}}
\def\PYG@tok@gs{\let\PYG@bf=\textbf}
\def\PYG@tok@gr{\def\PYG@tc##1{\textcolor[rgb]{1.00,0.00,0.00}{##1}}}
\def\PYG@tok@cm{\let\PYG@it=\textit\def\PYG@tc##1{\textcolor[rgb]{0.25,0.50,0.56}{##1}}}
\def\PYG@tok@vg{\def\PYG@tc##1{\textcolor[rgb]{0.73,0.38,0.84}{##1}}}
\def\PYG@tok@m{\def\PYG@tc##1{\textcolor[rgb]{0.13,0.50,0.31}{##1}}}
\def\PYG@tok@mh{\def\PYG@tc##1{\textcolor[rgb]{0.13,0.50,0.31}{##1}}}
\def\PYG@tok@cs{\def\PYG@tc##1{\textcolor[rgb]{0.25,0.50,0.56}{##1}}\def\PYG@bc##1{\colorbox[rgb]{1.00,0.94,0.94}{##1}}}
\def\PYG@tok@ge{\let\PYG@it=\textit}
\def\PYG@tok@vc{\def\PYG@tc##1{\textcolor[rgb]{0.73,0.38,0.84}{##1}}}
\def\PYG@tok@il{\def\PYG@tc##1{\textcolor[rgb]{0.13,0.50,0.31}{##1}}}
\def\PYG@tok@go{\def\PYG@tc##1{\textcolor[rgb]{0.19,0.19,0.19}{##1}}}
\def\PYG@tok@cp{\def\PYG@tc##1{\textcolor[rgb]{0.00,0.44,0.13}{##1}}}
\def\PYG@tok@gi{\def\PYG@tc##1{\textcolor[rgb]{0.00,0.63,0.00}{##1}}}
\def\PYG@tok@gh{\let\PYG@bf=\textbf\def\PYG@tc##1{\textcolor[rgb]{0.00,0.00,0.50}{##1}}}
\def\PYG@tok@ni{\let\PYG@bf=\textbf\def\PYG@tc##1{\textcolor[rgb]{0.84,0.33,0.22}{##1}}}
\def\PYG@tok@nl{\let\PYG@bf=\textbf\def\PYG@tc##1{\textcolor[rgb]{0.00,0.13,0.44}{##1}}}
\def\PYG@tok@nn{\let\PYG@bf=\textbf\def\PYG@tc##1{\textcolor[rgb]{0.05,0.52,0.71}{##1}}}
\def\PYG@tok@no{\def\PYG@tc##1{\textcolor[rgb]{0.38,0.68,0.84}{##1}}}
\def\PYG@tok@na{\def\PYG@tc##1{\textcolor[rgb]{0.25,0.44,0.63}{##1}}}
\def\PYG@tok@nb{\def\PYG@tc##1{\textcolor[rgb]{0.00,0.44,0.13}{##1}}}
\def\PYG@tok@nc{\let\PYG@bf=\textbf\def\PYG@tc##1{\textcolor[rgb]{0.05,0.52,0.71}{##1}}}
\def\PYG@tok@nd{\let\PYG@bf=\textbf\def\PYG@tc##1{\textcolor[rgb]{0.33,0.33,0.33}{##1}}}
\def\PYG@tok@ne{\def\PYG@tc##1{\textcolor[rgb]{0.00,0.44,0.13}{##1}}}
\def\PYG@tok@nf{\def\PYG@tc##1{\textcolor[rgb]{0.02,0.16,0.49}{##1}}}
\def\PYG@tok@si{\let\PYG@it=\textit\def\PYG@tc##1{\textcolor[rgb]{0.44,0.63,0.82}{##1}}}
\def\PYG@tok@s2{\def\PYG@tc##1{\textcolor[rgb]{0.25,0.44,0.63}{##1}}}
\def\PYG@tok@vi{\def\PYG@tc##1{\textcolor[rgb]{0.73,0.38,0.84}{##1}}}
\def\PYG@tok@nt{\let\PYG@bf=\textbf\def\PYG@tc##1{\textcolor[rgb]{0.02,0.16,0.45}{##1}}}
\def\PYG@tok@nv{\def\PYG@tc##1{\textcolor[rgb]{0.73,0.38,0.84}{##1}}}
\def\PYG@tok@s1{\def\PYG@tc##1{\textcolor[rgb]{0.25,0.44,0.63}{##1}}}
\def\PYG@tok@gp{\let\PYG@bf=\textbf\def\PYG@tc##1{\textcolor[rgb]{0.78,0.36,0.04}{##1}}}
\def\PYG@tok@sh{\def\PYG@tc##1{\textcolor[rgb]{0.25,0.44,0.63}{##1}}}
\def\PYG@tok@ow{\let\PYG@bf=\textbf\def\PYG@tc##1{\textcolor[rgb]{0.00,0.44,0.13}{##1}}}
\def\PYG@tok@sx{\def\PYG@tc##1{\textcolor[rgb]{0.78,0.36,0.04}{##1}}}
\def\PYG@tok@bp{\def\PYG@tc##1{\textcolor[rgb]{0.00,0.44,0.13}{##1}}}
\def\PYG@tok@c1{\let\PYG@it=\textit\def\PYG@tc##1{\textcolor[rgb]{0.25,0.50,0.56}{##1}}}
\def\PYG@tok@kc{\let\PYG@bf=\textbf\def\PYG@tc##1{\textcolor[rgb]{0.00,0.44,0.13}{##1}}}
\def\PYG@tok@c{\let\PYG@it=\textit\def\PYG@tc##1{\textcolor[rgb]{0.25,0.50,0.56}{##1}}}
\def\PYG@tok@mf{\def\PYG@tc##1{\textcolor[rgb]{0.13,0.50,0.31}{##1}}}
\def\PYG@tok@err{\def\PYG@bc##1{\fcolorbox[rgb]{1.00,0.00,0.00}{1,1,1}{##1}}}
\def\PYG@tok@kd{\let\PYG@bf=\textbf\def\PYG@tc##1{\textcolor[rgb]{0.00,0.44,0.13}{##1}}}
\def\PYG@tok@ss{\def\PYG@tc##1{\textcolor[rgb]{0.32,0.47,0.09}{##1}}}
\def\PYG@tok@sr{\def\PYG@tc##1{\textcolor[rgb]{0.14,0.33,0.53}{##1}}}
\def\PYG@tok@mo{\def\PYG@tc##1{\textcolor[rgb]{0.13,0.50,0.31}{##1}}}
\def\PYG@tok@mi{\def\PYG@tc##1{\textcolor[rgb]{0.13,0.50,0.31}{##1}}}
\def\PYG@tok@kn{\let\PYG@bf=\textbf\def\PYG@tc##1{\textcolor[rgb]{0.00,0.44,0.13}{##1}}}
\def\PYG@tok@o{\def\PYG@tc##1{\textcolor[rgb]{0.40,0.40,0.40}{##1}}}
\def\PYG@tok@kr{\let\PYG@bf=\textbf\def\PYG@tc##1{\textcolor[rgb]{0.00,0.44,0.13}{##1}}}
\def\PYG@tok@s{\def\PYG@tc##1{\textcolor[rgb]{0.25,0.44,0.63}{##1}}}
\def\PYG@tok@kp{\def\PYG@tc##1{\textcolor[rgb]{0.00,0.44,0.13}{##1}}}
\def\PYG@tok@w{\def\PYG@tc##1{\textcolor[rgb]{0.73,0.73,0.73}{##1}}}
\def\PYG@tok@kt{\def\PYG@tc##1{\textcolor[rgb]{0.56,0.13,0.00}{##1}}}
\def\PYG@tok@sc{\def\PYG@tc##1{\textcolor[rgb]{0.25,0.44,0.63}{##1}}}
\def\PYG@tok@sb{\def\PYG@tc##1{\textcolor[rgb]{0.25,0.44,0.63}{##1}}}
\def\PYG@tok@k{\let\PYG@bf=\textbf\def\PYG@tc##1{\textcolor[rgb]{0.00,0.44,0.13}{##1}}}
\def\PYG@tok@se{\let\PYG@bf=\textbf\def\PYG@tc##1{\textcolor[rgb]{0.25,0.44,0.63}{##1}}}
\def\PYG@tok@sd{\let\PYG@it=\textit\def\PYG@tc##1{\textcolor[rgb]{0.25,0.44,0.63}{##1}}}

\def\PYGZbs{\char`\\}
\def\PYGZus{\char`\_}
\def\PYGZob{\char`\{}
\def\PYGZcb{\char`\}}
\def\PYGZca{\char`\^}
\def\PYGZsh{\char`\#}
\def\PYGZpc{\char`\%}
\def\PYGZdl{\char`\$}
\def\PYGZti{\char`\~}
% for compatibility with earlier versions
\def\PYGZat{@}
\def\PYGZlb{[}
\def\PYGZrb{]}
\makeatother

\begin{document}

\maketitle
\tableofcontents
\phantomsection\label{main::doc}



\chapter{Proposal for PSOA2TPTP Translator}
\label{project-proposal/index:proposal-for-psoa2tptp-translator}\label{project-proposal/index::doc}\label{project-proposal/index:welcome-to-psoa2tptp-translator-s-documentation}

\section{Introduction}
\label{project-proposal/index:introduction}
Rule Language is a main type of formal languages for knowledge representation in
semntic web technologies.  PSOA RuleML {\hyperref[project-proposal/index:bol11a]{{[}Bol11a{]}}} is a rule language which
generalizes POSL {\hyperref[project-proposal/index:bol04]{{[}Bol04{]}}}, F-Logic and W3C's RIF-BLD languages
{\hyperref[project-proposal/index:boki10a]{{[}BoKi10a{]}}}, {\hyperref[project-proposal/index:boki10b]{{[}BoKi10b{]}}}.  IN PSOA RuleML, the positional-slotted,
object-applicative (PSOA) term is introduced as a generalization of the frame
term and the class membership term in RIF-BLD, as well as the positional-slotted
term in POSL language.  The planned two-part implementation of PSOA RuleML is
(1) convert PSOA RuleML's syntax into TPTP format, and (2) reads and executes
the TPTP with Vampire theorem prover \footnote{
\href{http://www.voronkov.com/vampire.cgi}{http://www.voronkov.com/vampire.cgi}
}.  In this project, we are going to
implement part (1) of the overarching project.


\section{Objectives}
\label{project-proposal/index:objectives}
This project is an important part of PSOA RuleML implementation.  The objectives
of the project is as follows:
\begin{quote}

1. Develop a translator to translate PSOA RuleML rule language documents into TPTP
format documents.  The input of the translator is a document conforming to
PSOA RuleML presentation syntax shown in {\hyperref[project-proposal/index:bol11a]{{[}Bol11a{]}}}.  The output of the
translator is a TPTP-FOF format document.

2. Create some test examples in PSOA RuleML syntax and their corresponding TPTP
format documents for testing the translator.  These examples can be further
extended to a complete PSOA test suite in the future.
\end{quote}


\section{Methodology}
\label{project-proposal/index:methodology}\label{project-proposal/index:index-0}

\subsection{Grammar Specification}
\label{project-proposal/index:grammar-specification}
In order to develop a translator from PSOA RuleML rule language into TPTP via
\emph{translation} (or transformation).  A \emph{program} specified in PSOA RuleML must be
recognized and parsed into a representation that lends itself to such a
\emph{transformation}.  This generally requires the use of a grammar specification
via specialized notation: Backus-Naur Form \footnote{
\href{http://en.wikipedia.org/wiki/Backus\%E2\%80\%93Naur\_Form}{http://en.wikipedia.org/wiki/Backus\%E2\%80\%93Naur\_Form}
}, Extended Backus-Naur Form \footnote{
\href{http://en.wikipedia.org/wiki/Extended\_Backus\%E2\%80\%93Naur\_Form}{http://en.wikipedia.org/wiki/Extended\_Backus\%E2\%80\%93Naur\_Form}
} or
more less known Parsing Expression Grammar \footnote{
\href{http://en.wikipedia.org/wiki/Parsing\_expression\_grammar}{http://en.wikipedia.org/wiki/Parsing\_expression\_grammar}
}.


\subsection{Compiler-compilers}
\label{project-proposal/index:compiler-compilers}
According to {\hyperref[project-proposal/index:wp11]{{[}Wp11{]}}}, a \textbf{compiler-compiler} or \textbf{compiler generator} is a
tool that creates a parser (and lexer), an interpreter/compiler from som form of
formal description of a language and machine.  The most prevalent form of
compiler-compiler is a \textbf{parser-generator} whose input is a grammar (like those
mentioned above) of a programming language and whose output is a collection of
source code for a parser (and lexer) often used as an initial (partial) set of components
for a compiler.

{\hyperref[project-proposal/index:wp11]{{[}Wp11{]}}} also mentions an open problem, that is the ``holy grail'' of
compiler-compilers such that a formal grammar along with a target platforms
instruction set may be given as inputs and the result would be a \emph{full} set of
compiler components capable of producing executable bytecode for machines
implementing the above instruction set.


\subsection{Abstract Syntax Trees}
\label{project-proposal/index:abstract-syntax-trees}
The result of applying a typical parser generator to a \emph{input program} (or
\emph{strings}) is a
\textbf{concrete syntax tree}, or \textbf{parse tree}: an ordered, rooted tree that
represents the syntactic structure of the \emph{input} according to some formally
specified grammar.  In a parse tree, the leaves are labeled by \emph{terminals} of the
grammar while the internal nodes are labeled as \emph{non-terminals} of the grammar.

According to {\hyperref[project-proposal/index:par07]{{[}Par07{]}}} and {\hyperref[project-proposal/index:wp11]{{[}Wp11{]}}} typical parser generators associate
pieces of executable code (written in a particular target language, e.g. Java, Python,
Ruby, etc.).  These pieces of code referred to as \textbf{actions} or \textbf{semantic
action routines} are executed when a particular rule of the grammar is applied
by the parser.  These routines may be used to specify the semantics of the
syntactic structure that is analyzed by the parser.

The result of applying the generated parser with
\textbf{semantic action routines} is an \textbf{abstract syntax tree}, or alternatively
executable code.  In this project we will concern ourselves with \textbf{abstract
syntax trees} only, such that the \emph{transformation/translation} we intend to
perform will be done against an \textbf{AST} to produce yet another \textbf{AST} that may
be used to yield \emph{executable} \textbf{TPTP-FOF} strings.


\section{Tools}
\label{project-proposal/index:tools}

\subsection{RuleML API}
\label{project-proposal/index:ruleml-api}
The primary purpose of the PSOA RuleML API is to provide abstract syntax
classes convenient for manipulating language constructs. Any Java sofware
using or processing PSOA RuleML will be able to benefit from this functionality.
The API will also provide two native parsers into the abstract syntax:
a presentation syntax parser to support simple text editor-based authoring,
and a parser for the XML serialisation, intended to support rule interchange.
{\hyperref[project-proposal/index:ria11]{{[}Ria11{]}}}


\subsection{ANTLR and AntlrWorks}
\label{project-proposal/index:antlr-and-antlrworks}
We will be working with ANTLR, a sophisticated parser-generator tool that is
popular and used to implement language interpreters and compilers.

ANTLRWorks is a graphical user interface development environment used to
facilitate the development of grammars (in \textbf{EBNF}) by providing the researcher
with tooling for testing/debugging the generated parser, running the parser
against various inputs, running rules in isolation against various inputs,
visualizing syntax diagrams of the grammar rules, visualizing the \textbf{concrete
syntax tree} produced by applying the generated parser against inputs.


\subsection{ANTLR Tree Description Language}
\label{project-proposal/index:antlr-tree-description-language}
As mentioned above parser generators typically support \textbf{action routines} to be
executed ``on-application'' of a given rule, ANTLR is no exception.  Additionally,
it supports a separate mechanism for constructing \textbf{trees} using the \code{-\textgreater{}}
operator and \textbf{ANTLR treed description language}.  Using ANTLR’s tree
description language, a tree is written like this:

\begin{Verbatim}[commandchars=@\[\]]
@textasciicircum[]( CLASS T @textasciicircum[](VARDEF int i) @textasciicircum[](VARDEF int j) @textasciicircum[](METHOD ...) ...)
\end{Verbatim}

This notation may also be used to recognize \emph{subtrees}, introducing a higher
level of abstraction, more suitable for mananging complexities of translating a
PSOA RuleML to another language such as TPTP-FOF.


\chapter{Overview}
\label{main:overview}
Knowledge representation is at the foundation of Semantic Web applications, using rule and ontology languages as the main kinds of formal languages. PSOA RuleML\textless{}\href{http://ruleml.org/\#PSOA}{http://ruleml.org/\#PSOA}\textgreater{} is a recently developed rule language which combines the ideas of relational (predicate-based) and object-oriented (frame-based) modeling. In order to support reasoning in PSOA RuleML, we are implementing a translator to map PSOA RuleML knowledge bases and queries to the TPTP\textless{}\href{http://www.cs.miami.edu/~tptp/}{http://www.cs.miami.edu/\textasciitilde{}tptp/}\textgreater{} format, as widely used for theorem provers. With this translator, reasoning in PSOA RuleML will be available using the VampirePrime\textless{}\href{http://www.vprover.org/index.cgi}{http://www.vprover.org/index.cgi}\textgreater{} prover. In our implementation, we use the Antlr3.0 parser generator tool to create a parser for PSOA RuleML and a tree parser to produce TPTP strings for a subset of the PSOA language. We have completed the Antlr lexer rules and grammar rules based on work by Alexandre. After parsing, an Antlr Abstract Syntax Tree (AST) is generated by the tree rewriting rules in the grammar file. Then, in the tree parser, the generated AST is parsed, based on the tree grammar rules, and TPTP strings are created. Finally, we run the TPTP-translated PSOA RuleML sample documents in VampirePrime to get the query results.


\chapter{Files}
\label{main:files}\begin{itemize}
\item {} 
Project Proposal Document

\item {} 
\href{http://dl.dropbox.com/u/46951970/Demo.zip}{Project Demo}

\end{itemize}

\begin{thebibliography}{BoKi10a}
\bibitem[Bol11a]{Bol11a}{\phantomsection\label{project-proposal/index:bol11a} 
Boley H., A RIF-Style Semantics for RuleML-Integrated Positional-Slotted, Object-Applicative Rules, RuleML Europe 2011, 194-211
}
\bibitem[Bol04]{Bol04}{\phantomsection\label{project-proposal/index:bol04} 
Boley H., POSL: An Integrated Positional-Slotted Language for Semantic Web Knowledge, \href{http://ruleml.org/submission/ruleml-shortation.html}{http://ruleml.org/submission/ruleml-shortation.html}
}
\bibitem[BoKi10a]{BoKi10a}{\phantomsection\label{project-proposal/index:boki10a} 
Boley H., M. Kifer, A Guide to the Basic Logic Dialect for Rule Interchange on the Web. IEEE Transactions on Knowledge and Data Engineering, 22(11):1593-1608
}
\bibitem[BoKi10b]{BoKi10b}{\phantomsection\label{project-proposal/index:boki10b} 
Boley H., M. Kifer, RIF Basic Logic Dialect, \href{http://www.w3.org/TR/rif-bld/}{http://www.w3.org/TR/rif-bld/}
}
\bibitem[Wp11]{Wp11}{\phantomsection\label{project-proposal/index:wp11} 
Compiler-Compiler, Wikipedia: The Free Encylopedia, \href{http://en.wikipedia.org/wiki/Parser\_generator}{http://en.wikipedia.org/wiki/Parser\_generator}
}
\bibitem[Par07]{Par07}{\phantomsection\label{project-proposal/index:par07} 
Parr T., The Definitive ANTLR Reference: Building Domain
Specific Languages, 2007, Pragmatic Programmer, USA.
}
\bibitem[Ria11]{Ria11}{\phantomsection\label{project-proposal/index:ria11} 
Skype conversation with Alex Riazanov
}
\end{thebibliography}



\renewcommand{\indexname}{Index}
\printindex
\end{document}
